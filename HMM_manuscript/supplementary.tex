\documentclass[11pt,oneside,letterpaper]{article}

% ---------- Packages (mirrors main manuscript) ----------
\usepackage{graphicx}
\DeclareGraphicsExtensions{.pdf,.png,.jpg}

\usepackage{color}
\usepackage{parskip}
\usepackage{float}
\usepackage[hyphens]{url}
\usepackage{xurl}
\usepackage[hidelinks,breaklinks]{hyperref}

\usepackage{geometry}
\geometry{textwidth=15cm}
\geometry{textheight=22cm}

\renewcommand{\topfraction}{0.85}
\renewcommand{\textfraction}{0.1}

\usepackage[labelfont=bf,labelsep=period,font=small]{caption}

% \usepackage{setspace}
% \doublespacing
% \captionsetup{labelfont=bf,labelsep=period,font=doublespacing}

\usepackage{cite}
% \renewcommand\citeleft{(}
% \renewcommand\citeright{)}
% \renewcommand\citeform[1]{\textsl{#1}}

\renewcommand\refname{\large References}
\makeatletter
\renewcommand{\@biblabel}[1]{\quad#1.}
\makeatother

\usepackage{authblk}
\renewcommand\Authands{ \& }
\renewcommand\Authfont{\normalsize \bf}
\renewcommand\Affilfont{\small \normalfont}
\makeatletter
\renewcommand\AB@affilsepx{, \protect\Affilfont}
\makeatother

\usepackage{amsmath}
\DeclareMathOperator*{\argmax}{arg\,max}
\usepackage{amssymb}
\usepackage{booktabs}

% ---------- Your notation macros (copied for consistency) ----------
\newcommand{\virus}{\mathbf{x}}
\newcommand{\serum}{\mathbf{y}}
\newcommand{\viruses}{\mathbf{X}}
\newcommand{\sera}{\mathbf{Y}}
\newcommand{\ve}{v}
\newcommand{\se}{s}
\newcommand{\ves}{\mathbf{v}}
\newcommand{\ses}{\mathbf{s}}
\newcommand{\point}{f_{\scriptscriptstyle \vert}}
\newcommand{\threshold}{f_{\textstyle \lrcorner}}
\newcommand{\interval}{f_{\sqcup}}
\newcommand{\mdssd}{\varphi}
\newcommand{\virussd}{\sigma_x}
\newcommand{\serumsd}{\sigma_y}
\newcommand{\drift}{\mu}
\newcommand{\tree}{\tau}
\newcommand{\vn}{n}
\newcommand{\sn}{k}
\newcommand{\normal}{\mathcal{N}}
\newcommand{\bwithin}{\beta_w}
\newcommand{\bsister}{\beta_s}
% NOTE: \bother macro in your snippet references \betZ_t (undefined). Keep or fix as needed.
% \newcommand{\bother}{\betZ_t}
\newcommand{\incclade}[1]{y_\mathrm{#1}}
\newcommand{\driftclade}[1]{x_\mathrm{#1}}
\newcommand{\States}{[M]}
\setlength{\arraycolsep}{2pt}
\newcommand{\smalltwomatrix}[2]{\scriptsize \Big( \begin{matrix} #1 \\ #2 \end{matrix} \Big)}
\newcommand{\smallfourmatrix}[4]{\scriptsize \Big( \begin{matrix} #1 & #2 \\ #3 & #4 \end{matrix} \Big)}
\newcommand{\twomatrix}[2]{\left( \begin{matrix} #1 \\ #2 \end{matrix} \right)}
\newcommand{\fourmatrix}[4]{\left( \begin{matrix} #1 & #2 \\ #3 & #4 \end{matrix} \right)}

% ---------- "S" prefix numbering ----------
% Sections, figures, tables, and equations will be labeled S1, S2, ...
\renewcommand{\thesection}{S\arabic{section}}
\renewcommand{\thesubsection}{S\arabic{section}.\arabic{subsection}}
\renewcommand{\thefigure}{S\arabic{figure}}
\renewcommand{\thetable}{S\arabic{table}}
\renewcommand{\theequation}{S\arabic{equation}}

% Optional: page numbers as S-1, S-2, ...
% \renewcommand{\thepage}{S\arabic{page}}

%%% TITLE %%%
\title{\vspace{1.0cm} \Large \bf
Supplementary Information for\\[3pt]
\textit{Hidden Markov Models Detect Pango Lineage Ancestry of Recombinant SARS-CoV-2 Sequences}
}

\author[1]{Nobuaki Masaki}
\author[2]{Trevor Bedford}

\affil[1]{Department of Biostatistics, University of Washington, Seattle, WA}
\affil[2]{Vaccine and Infectious Disease Division, Fred Hutch Cancer Center, Seattle, WA}

\date{} % No date

\begin{document}
\maketitle
\vspace{-0.5em}

% ---------- Optional front matter ----------
% \tableofcontents
% \listoffigures
% \listoftables

\section{Efficient forward algorithm}

We implemented an efficient version of the forward algorithm, reducing the time complexity of the induction step from $\mathcal{O}(M^2)$ to $\mathcal{O}(M)$, where $M$ is the number of unique Pango lineages. Using the notation from the main paper, we define,

\begin{align*}
    \alpha_t(i) = P(O_{1:t}=k_{1:t}, Z_t=i|\lambda,\epsilon),
\end{align*}

which are our forward probabilities. This represents the probability of the observed nucleotide sequence up to position $t$ and the ancestral Pango lineage being lineage $i$ at position $t$. 

In the induction step, we calculate the next time step for the forward probabilities. We have,

\begin{align*}
    \alpha_{t+1}(j) = \left(\sum_{i=1}^M \alpha_t(i)a_{ij}\right)b_{j,t+1}(k_{t+1}).
\end{align*}

Computing $\alpha_{t+1}(j)$ for one Pango lineage $j$ requires summing over $M$ lineages, which costs $\mathcal{O}(M)$. Thus, computing this for all Pango lineages costs $\mathcal{O}(M^2)$. 

In our transition matrix, we have equal diagonal entries and equal off-diagonal entries. Recall,

\begin{align*}
a_{ij} =
\begin{cases}
1-\lambda, & \text{if } i = j, \\
\frac{\lambda}{M - 1}, & \text{if } i \neq j.
\end{cases}
\end{align*}

Furthermore, we use the scaled version of the forward probabilities, meaning that $\sum_{i=1}^M \alpha_t(i) = 1$. Thus, we can rewrite the induction step as,

\begin{align*}
    \alpha_{t+1}(j) &= \left(\left(1-\alpha_t(j)\right)\frac{\lambda}{M-1} + \alpha_{t}(j)(1-\lambda)\right)b_{j,t+1}(k_{t+1}) \\
    &= \left(1-\lambda-\frac{\lambda}{M-1}\right)\alpha_t(j)+\frac{\lambda}{M-1} \\
    &= \left(1-\frac{M}{M-1}\lambda\right)\alpha_t(j)+\frac{\lambda}{M-1},
\end{align*}

which is constant time. Thus, computing this for all Pango lineages now costs $\mathcal{O}(M)$. 

% This Supplementary Information provides additional methods, figures, and tables supporting the main text.

% \section{Supplementary Methods}
% Numbered equations are prefixed with “S”. For example,
% \begin{equation}
% \label{eq:S1}
% \mathcal{L} = \prod_{t=1}^{T} p(y_t \mid x_t)\, p(x_t \mid x_{t-1}).
% \end{equation}

% \section{Supplementary Figures}
% \begin{figure}[H]
%   \centering
%   % \includegraphics[width=0.8\textwidth]{figs/supp_figure_S1.pdf}
%   \caption{\textbf{Title for Supplementary Figure S\thefigure.} A concise caption describing the figure contents, methods, and key findings.}
%   \label{fig:S1}
% \end{figure}

% \section{Supplementary Tables}
% \begin{table}[H]
%   \centering
%   \caption{\textbf{Title for Supplementary Table S\thetable.} A brief description of the table.}
%   \label{tab:S1}
%   \begin{tabular}{lcc}
%     \toprule
%     Item & Value 1 & Value 2 \\
%     \midrule
%     A    & 1.23    & 4.56    \\
%     B    & 7.89    & 0.12    \\
%     \bottomrule
%   \end{tabular}
% \end{table}

% \section{Supplementary References}
% Keep the same citation style as the main text
% If using a shared .bib file:
% \bibliographystyle{unsrt}
% \bibliography{references}
% Or, if journal requires a manual list:


\end{document}
